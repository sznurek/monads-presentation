\documentclass{article}[12pt]

\usepackage[utf8]{inputenc}
\usepackage{polski}
\usepackage[hidelinks]{hyperref}

\title{Krótko o wolnych monadach}
\author{Łukasz Dąbek}
\date{\today}

\begin{document}
\maketitle
Aby z funktora \texttt{m} uczynić monadę należy napisać parę funkcji:
\begin{itemize}
    \item \texttt{return :: a -> m a},
    \item \texttt{(>>=) :: m a -> (a -> m b) -> m b}.
\end{itemize}
Korzystając z faktu, że mamy do dyspozycji funkcję \texttt{fmap} możemy
równoważnie napisać inną parę funkcji:
\begin{itemize}
    \item \texttt{return :: a -> m a},
    \item \texttt{join :: m (m a) -> m a}.
\end{itemize}
Wolna monada to taka monada, która wykonuje minimalną pracę przy \texttt{join}ie.

Zaczniemy od definicji typu danych. Dla dowolnego funktora \texttt{f} definiujemy:
\begin{verbatim}
data Free f a = Pure a | Free (f (Free f a))
\end{verbatim}
W ramach rozgrzewki możemy napisać instancję funktora:
\begin{verbatim}
instance Functor f => Functor (Free f) where
    fmap f (Pure a) = Pure (f a)
    fmap f (Free c) = Free (fmap (fmap f) c)
\end{verbatim}

Zanim przejdziemy do implementacji interfejsu monadycznego rozważmy ,,kształt''
danych typu \texttt{Free f a} dla przykładowych funktorów:
\begin{itemize}
    \item \texttt{f = Maybe}: w tym wypadku struktura wygląda jak lista ,,bez danych'' na ,,consach''.
    \item \texttt{f = []}: w tym wypadku struktura wygląda jak drzewo.
    \item \texttt{f = (a,)}: w tym wypadku struktura wygląda jak regularna lista.
\end{itemize}

Z powyższych obserwacji możemy opisać kształt typu \texttt{Free f a} na potrzeby
naszych intuicji:
\begin{quotation}
Wolna monada nad funktorem f jest drzewem, które zawiera dane typu polimorficznego w
liściach i w każdym wierzchołku wewnętrznym zawiera pojemnik (funktor) dzieci.
\end{quotation}
Przykładowo struktura typu \texttt{Free [] Int} jest drzewem przechowującym
w liściach liczby całkowite.

Uzbrojeni w nowe intuicje możemy zinterpretować typ funkcji \texttt{join} dla
wolnych monad: jest to funkcja, która przetwarza drzewo drzew na normalne drzewo.
Jak to robi? Podczepiając w miejsce liści wspomniane drzewa!

Zobaczmy implementację interfejsu monadycznego dla wolnych monad:
\begin{verbatim}
instance Functor f => Monad (Free f) where
    return = Pure
    join (Pure a) = a
    join (Free c) = Free (fmap join c)
\end{verbatim}

Zauważmy, że przy implementacji funkcji \texttt{join} tylko pierwsze równanie
wykonuje jakąkolwiek realną pracę: drugie równanie wywołuje funkcję \texttt{join}
na swoich dzieciach.

Możemy jeszcze zinterpretować typ funkcji \texttt{(>>=)} w terminach drzew. Przypomnijmy
ten typ: \texttt{Free f a -> (a -> Free f b) -> Free f b}. Przekładając na język drzew:
funkcja \texttt{bind} bierze drzewo z liśćmi typu \texttt{a} i produkuje drzewo z liśćmi
typu \texttt{b}. Robi to przy pomocy funkcji podanej w drugim argumencie: jest to funkcja,
która dla każdego z liści wstawia na jego miejsce nowe drzewo.

Piotrek Polesiuk i Wojtek Jedynak wspomnieli mi, że wolne monady wyglądają bardzo podobnie
do monady podstawiającej zmienne. Faktycznie tak jest, a po szczegóły odsyłam do Dana
Piponiego\cite{subst}.

Więcej o wolnych monadach można poczytać na blogu Gabriela Gonzalesa\cite{free}.

\begin{thebibliography}{9}
    \bibitem{free}
        Gabriel Gonzalez,
        \emph{Why free monads matter}.
        \url{http://www.haskellforall.com/2012/06/you-could-have-invented-free-monads.html}.
    \bibitem{subst}
        Dan Piponi,
        \emph{Variable substitution gives a...}.
        \url{http://blog.sigfpe.com/2006/11/variable-substitution-gives.html}.
\end{thebibliography}

\end{document}
