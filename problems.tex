\documentclass{article}[12pt]

\usepackage[utf8]{inputenc}
\usepackage{polski}

\title{Zadania do prezentacji o monadach}
\author{Łukasz Dąbek}
\date{\today}

\begin{document}
\maketitle

Rozwiązania poniższych zadań należy wysłać na adres \texttt{sznurek@gmail.com}
do dnia \date{17.11.2013 r.} Na podany adres mailowy można również przesyłać
wszelkie pytania dotyczące zadań bądź prezentacji.

\section{Zadanie 1 - monadyczny interpreter}
\subsection{Treść zadania}
Zadanie składa się z dwóch części: napisaniu interpretera języka
\textsc{While} w stylu monadycznym a następnie rozszerzeniu go o instrukcję
\texttt{abort} oraz niedeterminizm.

Załóżmy, że mamy daną kategorię gramatyczną dla liczb naturalnych \texttt{<nat>}
oraz zmiennych \texttt{<var>}. Gramatyka języka \textsc{While} bez wspomnianych
wyżej rozszerzeń jest następująca:
\begin{verbatim}
<a-op>   ::= + | - | *
<b-op>   ::= and | or
<a-expr> ::= <nat> | <var> | <a-expr> <a-op> <a-expr>
<b-expr> ::= true | false | not <b-expr> | <b-expr> <b-op> <b-expr>

<stmt>   ::= skip | <var> := <a-expr> | <stmt>; <stmt> |
             if <b-expr> then <stmt> else <stmt> |
             while <b-expr> <stmt>
\end{verbatim}

Dla wyrażeń arytmetycznych i boolowskich należy napisać dwie
funkcje interpretujące:
\begin{verbatim}
evalA :: (Var -> Integer) -> Aexp -> Integer
evalB :: (Var -> Integer) -> Bexp -> Integer
\end{verbatim}

Dla instrukcji należy napisać interpreter w stylu monadycznym i przetestować
go dla monady identycznościowej:
\begin{verbatim}
eval :: Monad m => (Var -> Integer) -> Stmt -> m (Var -> Integer)
eval env Skip = return env
...
\end{verbatim}

Kolejnym krokiem jest rozszerzenie języka \textsc{While} o dwie instrukcje:
\begin{verbatim}
<stml> ::= (...) | abort <a-expr> | amb <stmt> <stmt>
\end{verbatim}
Intuicyjnie instrukcja \texttt{abort n} powinna przerwać działanie programu
z kodem błędu równym wartości wyrażenia \texttt{n}. Instrukcja \texttt{amb}
niedeterministycznie wybiera jedną z gałęzi obliczeń.

Aby zamodelować wspomniane efekty należy:
\begin{enumerate}
    \item Zdefiniować polimorficzny typ danych \texttt{M} (podpowiedź: powinien
        być kombinacją \texttt{Either} i listy).
    \item Dla typu \texttt{M} zdefiniować instancję \texttt{Functor} i \texttt{Monad}.
    \item Rozszerzyć wcześniej napisany interpreter o obsługę nowych instrukcji.
\end{enumerate}

\subsection{Ocenianie}
Rozwiązanie powinno składać się z dwóch osobnych plików źródłowych (po jednym
dla każdej części zadania). Na ocenę zadania składają się następujące kryteria,
w kolejności malejącego znaczenia:
\begin{enumerate}
    \item Wykonanie polecenia (tzn. interpretery napisane w innym stylu niż
        monadyczny nie będą oceniane).
    \item Poprawność interpertera względem semantyki poznanej na przedmiocie
        Programowanie M.
    \item Styl i czytelność kodu.
\end{enumerate}

\section{Zadanie 2 - monadyczny odkrywca}
\subsection{Treść zadania}
W poprzednim zadaniu szukaliśmy monady dla konkretnego zastosowania. Teraz
zrobimy odwrotnie: poszukamy instancji monady i zastosowania dla podanego
typu danych.

Zdefiniujmy nowy typ danych:
\begin{verbatim}
data Sel r a = Sel ((a -> r) -> a)
\end{verbatim}

Aby wykonać zadanie należy:
\begin{enumerate}
    \item Zdefiniować instancję klasy typów \texttt{Functor} dla typu \texttt{Sel}.
        Upewnij się, że Twoja definicja spełnia odpowiednie prawa.
    \item Zdefiniować instację klasy typów \texttt{Monad} dla typu \texttt{Sel}.
        Ponownie, upewnij się, że Twoja implementacja spełnia prawa monadyczne.
        \footnote{W przeciwnym przypadku do Twoich drzwi może zapukać Międzygwiezdna
        Policja Monadyczna.}
    \item (Najważniejsze i najtrudniejsze) Znaleźć zastosowanie dla właśnie zdefiniowanej
        monady.
\end{enumerate}

Ponieważ ostatni krok jest trudny i niedospecyfikowany poniżej znajduje się kilka
wskazówek:
\begin{enumerate}
    \item Nazwa typu danych została wybrana nieprzypadkowo.
    \item Rozważ przypadki, gdy \texttt{r = Bool} i \texttt{r = Int}. Następnie
        postaraj się zinterpretować typ \texttt{a -> Bool} jako predykat i
        \texttt{a -> Int} jako funkcję oceny.
    \item Monada \texttt{Sel} ma związek z monadą kontynuacyjną (widzisz podobieństwo
        typów?).
    \item Monada \texttt{Sel} może pomóc napisać Ci algorytm brutalny albo wygrać
        w Nima!
    \item Wspomóż się haskellową notacją \texttt{do} aby wyrobić sobie intuicje.
\end{enumerate}

\subsection{Ocenianie}
Na ocenę zadania składa się (z równą wagą):
\begin{enumerate}
    \item Napisanie poprawnych instancji klas \texttt{Functor} i \texttt{Monad}
        dla typu \texttt{Sel}.
    \item Znalezienie zastosowania dla odkrytej monady i zilustrowanie tego
        krótkim przykładem.
\end{enumerate}
Przewidziane są bonusowe punkty za styl.

\end{document}
